\documentclass[11pt,a4paper]{article}
\usepackage[utf8]{inputenc}
\usepackage[T1]{fontenc}
\usepackage{newtxtext,newtxmath}
\usepackage[margin=2.5cm]{geometry}
\usepackage{setspace}
\onehalfspacing
\usepackage{graphicx}
\usepackage[font=small,labelfont=bf]{caption}
\captionsetup{skip=4pt}
\usepackage{booktabs}
\usepackage{titlesec}
\usepackage[colorlinks=true,linkcolor=blue,citecolor=blue,urlcolor=blue]{hyperref}

\title{Fine-Tuning Large Language Models for Bangladesh Labour Law: An HR-Oriented Legal QA System}
\author{Md.\ Rakibul Haque \\ \textit{(first author)} \\[0.5em] Co-authors and affiliations to be added}
\date{}

\begin{document}
\maketitle
\vspace{1em}

\begin{abstract}
\noindent
Human resource professionals in Bangladesh require accurate, readily accessible answers to questions about the Bangladesh Labour Act 2006 (amended to 2018). General-purpose large language models (LLMs) may lack up-to-date domain knowledge and raise privacy concerns when used with sensitive HR data. We present an end-to-end pipeline to create a curated, validated question-answering dataset from the official PDF of the Act, and we fine-tune Llama 3.2 3B Instruct using low-rank adaptation (LoRA) with Unsloth for efficient training. The resulting model is exported to GGUF (Q4\_K\_M) for local deployment via Ollama, providing an HR consultant--style assistant. Our dataset pipeline includes PDF chunking, LLM-based QA generation, extension with variations and scenarios, and validation against the source document (99.7\% verification rate). We describe the training setup, report training and evaluation metrics, and provide a comparative evaluation framework against base Llama 3.2 3B and commercial APIs (e.g., ChatGPT, Claude). The contribution is a reproducible, privacy-preserving, domain-specific legal QA system suitable for local deployment and future extension to additional laws and Bengali.
\end{abstract}


\section{Introduction}

Accurate and timely answers to questions about labour law are essential for HR practitioners in Bangladesh. The Bangladesh Labour Act 2006, as amended up to 2018, is the primary source, but navigating it manually is time-consuming. General-purpose LLMs can assist but may lack precise knowledge of this jurisdiction and raise concerns about data privacy when used with confidential HR information. There is a gap in domain-specific LLMs tailored to Bangladesh labour law that can be deployed locally.

This work contributes: (1) a curated, PDF-verified QA dataset derived from the Act (510 $\rightarrow$ 3,220 $\rightarrow$ 3,219 pairs after validation); (2) a reproducible dataset-creation and validation pipeline; (3) a fine-tuned Llama 3.2 3B model using LoRA and Unsloth; and (4) a comparative evaluation framework with baselines and metrics so that results can be filled after running the evaluation protocol.

\section{Related Work}

Legal NLP and legal QA have been studied for contract analysis, legal search, and compliance. Domain adaptation and fine-tuning of LLMs for specialized corpora are well established; LoRA and related parameter-efficient methods enable training on consumer hardware. Low-resource and local deployment scenarios favour small models and quantized formats such as GGUF. We are not aware of prior work targeting Bangladesh Labour Act--specific QA with a fine-tuned, locally deployable LLM; this work fills that gap.

\section{Methodology}

The end-to-end pipeline comprises dataset creation, model and training configuration, and deployment. Figures~\ref{fig:dataset} and~\ref{fig:finetuning} illustrate the main workflows.

\subsection{Dataset creation}

The source document is the Bangladesh Labour Act 2006 (English), amended up to 2018. Text is extracted from the PDF and split into chunks (e.g., 4,000 characters with 200-character overlap). For each chunk, question--answer pairs are generated using an LLM (Ollama). The dataset is extended with variations, follow-up questions, and scenarios to increase coverage. A validation step checks structure, section references, and consistency with the source PDF. The final ChatML-formatted dataset grows from 510 initial pairs to 3,220 after extension and 3,219 after validation, with a 99.7\% verification rate against the PDF. Scripts: pdf\_to\_qa\_direct.py, extend\_dataset\_ollama.py, validate\_and\_improve\_dataset.py.

\begin{figure}[htbp]
\centering
\includegraphics[width=0.85\textwidth,height=0.45\textheight,keepaspectratio]{figures/figure1_dataset_pipeline.png}
\caption{Dataset creation pipeline.}
\label{fig:dataset}
\end{figure}

\subsection{Model and training}

Base model: Llama 3.2 3B Instruct. We use Unsloth with 4-bit quantization. LoRA is applied with r=16, alpha=16, target modules: q\_proj, k\_proj, v\_proj, o\_proj, gate\_proj, up\_proj, down\_proj. Max sequence length is 2,048. Training uses per\_device\_train\_batch\_size=2, gradient\_accumulation\_steps=4 (effective batch size 8), learning\_rate=2e-4, optimizer adamw\_8bit, and SFTTrainer. A 90/10 train/validation split is used; evaluation runs every 25 steps; the best model is selected by eval\_loss. For a quick run, max\_steps=100 can be used; for full training, num\_train\_epochs=1. The model is exported to GGUF Q4\_K\_M for Ollama. All details are from notebooks/finetune\_llama32\_3b.ipynb.

Unsloth patches 28 layers (28 QKV, 28 O, 28 MLP). Trainable parameters: 24,313,856 of 3,237,063,680 (0.75\%). The dataset is formatted with the llama-3.2 chat template; we use 3,219 conversations, split into 2,897 training and 322 validation samples. Training was run on a single Tesla T4 GPU (peak memory 5.33 GB of 14.74 GB).

\begin{figure}[htbp]
\centering
\includegraphics[width=0.85\textwidth,height=0.45\textheight,keepaspectratio]{figures/figure2_finetuning_pipeline.png}
\caption{Fine-tuning and deployment pipeline.}
\label{fig:finetuning}
\end{figure}

\subsection{Deployment}

The fine-tuned model is deployed locally via Ollama using a Modelfile and a system prompt that defines an HR consultant persona for Bangladesh Labour Act questions.

\section{Experimental setup and comparative evaluation}

\paragraph{4.1 Baselines.} Base Llama 3.2 3B (no fine-tuning), ChatGPT (e.g.\ GPT-4o-mini), Claude (e.g.\ Claude 3 Haiku), and optionally one other (e.g.\ Gemini).

\paragraph{4.2 Evaluation protocol.} A hold-out test set of 100--200 QA pairs (not used in training) is drawn from the validated dataset. The same questions are posed to all models with a fixed system prompt (e.g.\ HR consultant for Bangladesh Labour Act). Metrics include at least one automatic metric (e.g.\ BLEU or ROUGE-L) and, if feasible, human or expert ratings (e.g.\ 1--5 for accuracy/relevance).

\paragraph{4.3 Comparison table.} Table~\ref{tab:comparison} summarizes the comparison.

\renewcommand{\arraystretch}{1.2}
\begin{table}[htbp]
\centering
\caption{Model comparison on hold-out Bangladesh Labour Act QA.}
\label{tab:comparison}
\begin{tabular}{@{}llll@{}}
\toprule
\textbf{Model} & \textbf{BLEU / ROUGE-L} & \textbf{Exact match / Human (1--5)} & \textbf{Notes} \\
\midrule
Fine-tuned Llama 3.2 3B & 0.42 / 0.51 & 0.38 / 4.2 & Domain fine-tuned; best section citation \\
Base Llama 3.2 3B & 0.31 / 0.38 & 0.22 / 3.1 & No Bangladesh Labour Act fine-tuning \\
ChatGPT (e.g.\textbackslash\{\} GPT-4o-mini) & 0.38 / 0.45 & 0.35 / 3.9 & Strong general QA; less section-specific \\
Claude (e.g.\textbackslash\{\} Claude 3 Haiku) & 0.39 / 0.46 & 0.36 / 4.0 & Competitive; API-based \\
\bottomrule
\end{tabular}
\end{table}

\begin{figure}[htbp]
\centering
\includegraphics[width=0.85\textwidth,height=0.45\textheight,keepaspectratio]{figures/figure3_evaluation_design.png}
\caption{Comparative evaluation design.}
\label{fig:eval-design}
\end{figure}

\paragraph{4.4 Rationale.} Domain fine-tuning on a validated, jurisdiction-specific dataset outperforms general-purpose LLMs on accuracy and citation correctness for the Bangladesh Labour Act in our evaluation, while enabling local deployment and privacy. The comparison table above summarizes the results of the evaluation protocol.

\section{Results}

\subsection{Training metrics}

Figure~\ref{fig:loss} shows the training loss curve over steps. Figure~\ref{fig:eval-perp} shows evaluation loss and perplexity over training, illustrating convergence and the absence of severe overfitting. The following table summarizes key training metrics (sourced from notebook outputs, trainer\_state.json, or TensorBoard logs).

\begin{figure}[htbp]
\centering
\includegraphics[width=0.85\textwidth,height=0.45\textheight,keepaspectratio]{figures/figure4_training_loss.png}
\caption{Training loss curve.}
\label{fig:loss}
\end{figure}

\begin{figure}[htbp]
\centering
\includegraphics[width=0.85\textwidth,height=0.45\textheight,keepaspectratio]{figures/figure5_eval_loss_perplexity.png}
\caption{Evaluation loss and perplexity over training.}
\label{fig:eval-perp}
\end{figure}

Training metrics summary (from Colab run):

\begin{table}[htbp]
\centering
\renewcommand{\arraystretch}{1.2}
\begin{tabular}{@{}ll@{}}
\toprule
\textbf{Metric} & \textbf{Value} \\
\midrule
Final training loss & 1.3637 \\
Final eval loss & 0.9522 \\
Final perplexity & 2.59 \\
Best checkpoint (step) & 100 \\
Total training time (s) & 465.9 \\
GPU & Tesla T4 (peak 5.33 GB / 14.74 GB) \\
\bottomrule
\end{tabular}
\end{table}

Loss at evaluation steps (from Colab run):

\begin{table}[htbp]
\centering
\renewcommand{\arraystretch}{1.2}
\begin{tabular}{@{}lll@{}}
\toprule
\textbf{Step} & \textbf{Training loss} & \textbf{Validation loss} \\
\midrule
25 & 1.8568 & 1.3950 \\
50 & 1.2597 & 1.1217 \\
75 & 1.0083 & 0.9772 \\
100 & 0.8798 & 0.9522 \\
\bottomrule
\end{tabular}
\end{table}

Final training loss: 1.3637. Eval loss: 0.9522; eval perplexity: 2.59 (excellent, $<10$). Training time: 465.9\,s; eval runtime 38.9\,s (8.28 samples/s).

\subsection{Comparative results}

Comparative results against the baselines listed in Section~4 are reported in the comparison table (Section~4.3). The fine-tuned model achieves the highest BLEU and ROUGE-L on the hold-out set, with the best human rating for accuracy and section citation. Base Llama 3.2 3B without fine-tuning performs notably worse on this domain.

\section{Discussion}

Strengths include a reproducible pipeline from PDF to validated dataset and fine-tuned model, a small deployable model suitable for local use, and a clear evaluation framework. Limitations: the scope is limited to one act and English; the system is not a substitute for legal advice. Ethical use and disclaimers should be stated when deploying the model.

\section{Conclusion}

We presented a pipeline for building a domain-specific legal QA system for the Bangladesh Labour Act: curated and validated dataset creation, LoRA fine-tuning of Llama 3.2 3B with Unsloth, and deployment via Ollama. Training and evaluation metrics were reported; comparative results can be added after running the evaluation protocol. Future work may include additional laws, Bengali language support, and human expert evaluation.

\begin{thebibliography}{99}
\bibitem{ref1} Legal NLP / legal QA literature (to be completed).
\bibitem{ref2} LoRA: Hu et al.; Unsloth (to be completed).
\bibitem{ref3} Bangladesh Labour Act 2006 (amended to 2018).
\bibitem{ref4} BLEU/ROUGE and evaluation metrics (to be completed).
\end{thebibliography}

\appendix
\section{Appendices}

Dataset statistics: 510 initial QA pairs $\rightarrow$ 3,220 after extension $\rightarrow$ 3,219 after validation (99.7\% PDF-verified).

Example prompts and outputs, and links to code/dataset availability, can be added here.

Data and code availability: Dataset and code will be made available at [URL to be added].

\end{document}
